% Introducción
\chapter*{Introduction} % Main chapter title

\label{Intro} % For referencing the chapter elsewhere, use \ref{Chapter1} 

\lhead{\emph{Introduction}} % This is for the header on each page - perhaps a shortened title

%----------------------------------------------------------------------------------------
%----------------------------------------------------------------------------------------
%	INTRODUCTION
%----------------------------------------------------------------------------------------

The curve graph $\Gamma(S)$ associated to a surface $S$ appears naturally in the study of $Mod(S)$, the mapping class group of $S$, which is a central subject in contemporary mathematical research. We are interested in a rigidity concept of this graph; in general, the idea behind rigidity phenomena is to describe morphisms among objects using their structure.

The folkloric version of rigidity in the $Mod(S)$ context is that if we consider $X$ and $Y$, under suitable conditions, then every homomorphism $Mod(X) \to Mod(Y)$ will be induced by a manipulation of the underlying surfaces.

Ivanov sketched in \cite[Ivanov 97]{celebratedIvanov} the proof that every automorphism of $C(S)$ (the flag complex of $\Gamma(S)$), is induced by a self-homeomorphism of $S$. This argument is the favorite in the literature due to its simplicity and resemblance to proofs of other rigidity results.

A research line lead by Aramayona and Leininger propose the use of \textit{rigid sets}, which can be interpreted as a subset \textit{that allows to extend a local notion of rigidity to a global one}. In the aim of finding large rigid sets, in \cite[Aramayona, Leininger 16]{finiteRigidSetsJA} there's the proof of the existence of an increasing sequence of finite rigid sets that exhaust the curve graph. For this, they proposed a method called \textbf{rigid expansions}.

Rigidity in graphs is, regardless of its interpretation in the curve graph, an interesting phenomenon by it self. Due to the discrete nature of rigid expansions is reasonable to seek for a probabilistic approach; our goal is to address this particular path.

We want to answer the rather vague question: \textit{How \textbf{common} is rigidity in graphs}, specifically by answering \textit{how rigid expansions \textbf{usually} behave}. Also, the aim of the thesis is to review the feasibility of \textit{studying the curve complex of a surface from a probabilistic point of view.}

Probabilistic models give formal meaning to words like "common" or "usually", we study rigidity phenomenon in this context and analyze the conditions under which these models fit the known properties of the curve graph.

In chapter one, we motivate the study of the curve graph and review the most important properties of it. Then, we introduce rigidity within the context of Graph theory.

In the second chapter, we propose the study of rigidity from the stochastic point of view through the Radó graph and the Erdös-Rényi model. In the aim to study the curve graph of a surface with a simple model, we justify that the genus of the surface cannot be finite. Thus, we end up with an asymptotic probabilistic analogue to the result due to Bering and Gaster, which asserts that the Radó graph embeds into the curve graph $C(S)$ of a surface $S$ if and only if $S$ has infinite genus.

Finally, we made a computational implementation of the algorithm to do rigid expansions. With the corresponding optimizations that the method require, we are able to take a closer look to rigidity phenomena.